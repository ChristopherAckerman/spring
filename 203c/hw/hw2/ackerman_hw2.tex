\documentclass[dvipsnames]{article}
\usepackage{amsmath,amsthm,amssymb}
\usepackage{graphicx}
\usepackage{hyperref}
\usepackage{textcomp}
% \usepackage{dsfont}
\usepackage{tabularx}
\usepackage{tikz}
\usepackage{physics}
\usepackage{changepage}% http://ctan.org/pkg/changepage
\usetikzlibrary{scopes,calc,arrows}
\usepackage{setspace}
\usepackage[makeroom]{cancel}
\usepackage{enumitem}
\usepackage[margin=1in]{geometry}
\usepackage[T1]{fontenc}
\usepackage[utf8]{inputenc}
\usepackage{tabularx,ragged2e,booktabs,caption}
\usepackage{wrapfig,lipsum,booktabs}
\usepackage{hanging}
\usepackage{multicol}
\usepackage{multirow}
\usepackage{blindtext}
\usepackage{booktabs}
\usepackage{color}
\usepackage{dcolumn}
% \usepackage{minted}
% \definecolor{light}{rgb}{0.35, 0.35, 0.35}
% \def\light#1{{\color{light}#1}}
\usepackage{mathtools}
\newcommand{\p}{\mathbb{P}}
\newcommand{\E}{\mathbb{E}}
\newcommand{\R}{\mathbb{R}}
\newcommand{\Var}{\operatorname{Var}}
\newcommand{\gr}{\textcolor{ForestGreen}}
\newcommand{\rd}{\textcolor{red}}


\usepackage[space]{grffile}

\title{203C HW1}
\author{Christopher Ackerman}
\date{\today}

\begin{document}
\maketitle

\section{Question 1}
\begin{enumerate}
\item No, this vector isn't identified. In order to identify the entire vector we need to identify each element. We can't identify $h$ within $H$ because we don't see observations on the entire domain $\R$, so we can't identify the vector.

\item Nothing is identified in this case. Denote true values with a $\ast$, and consider
  \begin{align*}
    \tilde{h}(x) &= c h^\ast(x)\\
    \tilde{\mu} &= c \mu^\ast \\
    \tilde{\sigma}^2 &= c^2 \sigma^{\ast 2}\\
    c &> 0\\
    \p(Y = 1 \mid X = x; \tilde{\mu} \tilde{\sigma}^, \tilde{h}) &= \Phi \left(\frac{\tilde{h}(x) - \tilde{\mu}}{\sqrt{\tilde{\sigma}^2}}\right)\\
                 &= \Phi \left(\frac{h^\ast(x) - \mu^\ast}{\sqrt{\sigma^{\ast 2}}}\right)\\
    &= \p(Y = 1 \mid X= x; \mu^\ast, \sigma^{\ast 2}, h^\ast)
  \end{align*}
\item With this normalization, we can write
  \begin{align*}
    \p(Y = 1 \mid X = x) &= \Phi ( h(x)) \\
    \implies h(x) &= \Phi^{-1}(\p(Y = 1 \mid X = x))
  \end{align*}
  Now $h(x)$ is identified in $H_x$, but it is not identified in $H$ because there are infinitely many points in $\R$ (e.g. $x = 6$) where the value of $h(x)$ could be different.
\item Plugging in this functional form to the result in the last part,
  \begin{align*}
    \alpha + \beta x &= \Phi^{-1}(\p (Y = 1 \mid X = x))
  \end{align*}
  We can pin down values for $\alpha$ and $\beta$ as
  \begin{align*}
    \alpha &= \E(Y \mid X = 0)\\
    \beta &= \E(Y \mid X = 1) - \E(Y \mid X = 0)
  \end{align*}
  Since $x= 0$ and $x = 1$ are in the support of both $H$ and $H_X$, we have identification within both sets.
\end{enumerate}
\section{Question 2}
\begin{enumerate}
\item
  \begin{align*}
    \p(Y = 1 \mid X = x) &= \p(\varepsilon \le x^\prime \beta \mid X = x)\\
                         &= \p \left(z \le \frac{x^\prime \beta - \mu}{\sqrt{s^2(x)}} \mid X = x \right)\\
    &= \Phi \left(\frac{x^\prime \beta - \mu}{\sqrt{s^2(x)}}\right)
  \end{align*}

\item No, $\beta$ is not identified. Take
  \begin{align*}
    \tilde{\beta} &\equiv c \beta^\ast \\
    \tilde{\mu} &\equiv c \mu^\ast \\
    \tilde{s}^2(x) &\equiv c^2 s^{\ast 2}(x)\\
    \p(Y = 1 \mid X = x; \tilde{\beta}, \tilde{\mu}, \tilde{s}) &= \Phi \left(\frac{x^\prime \tilde{\beta} - \tilde{\mu}}{\sqrt{\tilde{s}^2(x)}}\right)\\
                  &= \Phi \left(\frac{x^\prime \beta^\ast - \mu^\ast}{\sqrt{s^{\ast 2}(x)}}\right)\\
    &= \p (Y = 1 \mid X = x; \beta^\ast, \mu^\ast, s^{\ast 2})
  \end{align*}
\item We can now write the conditional expectation as
  \[
\p(Y = 1 \mid x = x) = \Phi \left(\frac{x_1 + x_2 \beta - \mu}{s^2(x_3)}\right)
\]
We can identify $\beta_2$ by differentiation with respect to $x_2$, and we can identify $\sqrt{s^2(x_3)} = |s(x_3)$ by differentiating with respect to $x_1$. So $\beta_2$ is identified, and $s$ is identified if we're willing to assume that $s(x_3) \ge 0$.

\item Yes; we still have identification for both $s$ and $\beta_2$ since the argument in (c) does not depend on the value of $s(0)$.

\item They are both identified. To see that $F_\varepsilon$ is identified, define some $\tilde{F}_\varepsilon$ such that
  \begin{align*}
    F_\varepsilon^\ast (x_1 + v^\ast (x_2)) &= \tilde{F}_\varepsilon (x_1 + \tilde{v}(x_2))\quad \forall x\\
    \intertext{If we look at $x_2 = 0$, }
    F_\varepsilon^\ast (x_1) &= \tilde{F}_\varepsilon (x_1),\\
    \intertext{so $F_\varepsilon$ is identified. Now if we look at $x_1 = 0$,}
    F_\varepsilon (v^\ast(x_2)) &= F_\varepsilon(\tilde{v}(x_2)) \quad \forall x
  \end{align*}
  $F_\varepsilon$ is strictly increasing so we can take its inverse, and therefore $v$ is identified.

  \item The answer is still the same since we have the normalization/known value $h(\overline{x} = \alpha$. We would only have to change the argument if we didn't know this value.
\end{enumerate}
\section{Question 3}
\begin{enumerate}
\item
  \begin{align*}
    \E(Y \mid X = x) &= \p (Y = 1 \mid X = x) - (1 - \p(Y = 1 \mid X = x))\\
    &= 2 \p(Y = 1 \mid X = x) - 1
  \end{align*}
\item
  \begin{align*}
    \p(Y = 1 \mid X = x) &= \p(\varepsilon \le v(x_1) - v(x_2) \mid X = x)\\
                         &= F_{\varepsilon \mid X = x} (v(x_1) - v(x_2))\\
    v(x_1) - v(x_2) &= t > 0 \implies \\
    \p(Y = 1 \mid X = x) &= F_{\varepsilon \mid X = x}(t)\\
                         &> F_{\varepsilon \mid X = x}(0)\\
                         &= \frac{1}{2}\\
    v(x_1) - v(x_2) & =t < 0 \implies \\
    \p(Y = 1 \mid X =x) &= F_{\varepsilon \mid X = x} (t)\\
                         &< F_{\varepsilon \mid X = x} (0)\\
                         &= \frac{1}{2}\\
    \therefore \p(Y = 1 \mid X_1 = x_1, X_2 = x_2) &\ge \frac{1}{2} \iff v(x_1) \ge v(x_2)
  \end{align*}
\item
  We can define level sets at $V(\overline{x})$ as
  \begin{align*}
    S(V)(\overline{x}) &= \{x \mid V(x) \ge V(\overline{x})\}\\
                       &= \left\{x \mid \p(Y = 1 \mid X_1 = \overline{x}, X_2 = x) \ge \frac{1}{2}\right\}\\
    S^\prime(V) (\overline{x}) &= \{x \mid V(x) \le V(\overline{x})\}\\
    &= \left\{x \mid \p (Y = 1 \mid X_1 = x, X_2 = \overline{x}) \le \frac{1}{2} \right\}
  \end{align*}
  $V$ is identified when no function in $S$ or $S^\prime$ is an increasing function of another function in that set, and when $V(\overline{x}) = 0$ for some $\overline{x}$.

  \item We can only identify the value of $v$ at a single point $t = v(x_1) - v(x_2)$, so $F_{\varepsilon \mid X = x} (t)$ is not identified.
\end{enumerate}
\end{document}
