\documentclass{article}
\usepackage{amsmath,amsthm,amssymb}
\usepackage{graphicx}
\usepackage{hyperref}
\usepackage{textcomp}
% \usepackage{dsfont}
\usepackage{tabularx}
\usepackage{tikz}
\usepackage{physics}
\usepackage{changepage}% http://ctan.org/pkg/changepage
\usetikzlibrary{scopes,calc,arrows}
\usepackage{setspace}
\usepackage[makeroom]{cancel}
\usepackage{enumitem}
\usepackage[margin=1in]{geometry}
\usepackage[T1]{fontenc}
\usepackage[utf8]{inputenc}
\usepackage{tabularx,ragged2e,booktabs,caption}
\usepackage{wrapfig,lipsum,booktabs}
\usepackage{hanging}
\usepackage{multicol}
\usepackage{multirow}
\usepackage{blindtext}
\usepackage{booktabs}
\usepackage{color}
\usepackage{dcolumn}
% \usepackage{minted}
\definecolor{light}{rgb}{0.35, 0.35, 0.35}
\def\light#1{{\color{light}#1}}
\usepackage{mathtools}
\newcommand{\p}{\mathbb{P}}
\newcommand{\E}{\mathbb{E}}
\newcommand{\Var}{\operatorname{Var}}

\title{Econ 203C HW1, Spring 2021}
% \institution{Econ 202C}
\author{Chris Ackerman\thanks{I worked on this homework with Paige Pearcy and Luna Shen.}}
\date{\today}

\begin{document}
\maketitle
\newpage
\section*{Question 1}
\begin{enumerate}
\item 
  \begin{align*}
    F_{Y \mid X = x} (y, x) &= \p (Y \le y \mid X = x)\\
                            &= \p ( \varepsilon \le y - g(x) \mid X = x)\\
                            &= \p \left(\frac{\varepsilon - \mu(x)}{\sqrt{\sigma^2(x)}} \le \frac{y - (g(x) + \mu (x))}{\sqrt{\sigma^2(x)}} \mid X = x\right)\\
                            &= \Phi \left(\frac{y - (g(x) + \mu (x))}{\sqrt{\sigma^2(x)}}\right)\\
    Y \mid X &\sim \mathcal{N} (g(x) + \mu (x), \sigma^2 (x))
  \end{align*}
\item
  These results follow from properties of the normal distribution.
  \begin{align*}
    \E[Y \mid X = x] &= g(x) + \mu (x)\\
    \Var (Y \mid X = x) &= \sigma(x)^2
  \end{align*}
  \item The distribution function for this model must be a normal distribution whose first and second moments match the values from the previous question. In this case,
    \begin{align*}
      m_1(x) &= \int y dF_{Y \mid X = x} (y, x) \tag{first moment}\\
             &= g(x) + \mu (x)\\
      m_2(x) &= \int y^2 dF_{Y \mid X = x}(y, x) \tag{second moment}\\
      &= \sigma^2(x) + (g(x) + \mu(x))^2
    \end{align*}
    
  \item $\mu (x)$ is not identified. To see this, define
    \begin{align*}
      g^\ast (x) &\equiv g(x) - c\\
      \mu^\ast (x) &\equiv \mu (x) + c\\
      \implies g^\ast(x) + \mu^\ast(x) &= g(x) + \mu(x).
    \end{align*}
    $\sigma^2(x) = \Var(Y \mid X = x)$ is identified, because we observe $Y$ and $X$.
\end{enumerate}
\section*{Question 2}
\begin{enumerate}
\item
  \begin{align*}
    F_{Y \mid X = x} (y, x) &= \p(Y \le y \mid X = x)\\
                            &= \p(\varepsilon \le y - \alpha^\ast \cdot g^\ast (x) \mid X = x)\\
                            &= \p \left(\frac{\varepsilon - \mu^\ast}{\sqrt{(\sigma^\ast)^2}} \le \frac{y - (\alpha^\ast \cdot g^\ast (x) + \mu^\ast)}{\sqrt{(\sigma^\ast)^2}} \mid X = x\right)\\
    &= \Phi \left(\frac{y - (\alpha^\ast \cdot g^\ast(x) + \mu^\ast)}{\sqrt{(\sigma^\ast)^2}}\right)
  \end{align*}
\item $g^\ast$ is not identified within the class of continuous functions. If it were identified,
  \[
\E[Y \mid X = x, \alpha^\ast, g^\ast, \mu^\ast, \sigma^\ast] = \E[Y \mid X =x, \tilde{\alpha}, \tilde{\mu}, \tilde{g}, \tilde{\sigma}] \implies \alpha^\ast g^\ast (x) + \mu^\ast = \tilde{\alpha} \tilde{g} (x) + \tilde{\mu}.
\]
To find a counterexample, take
\begin{align*}
  \tilde{g}(x) &= g^\ast (x) + c\\
  \tilde{\alpha} &= \alpha^\ast\\
  \tilde{\mu} &= \mu^\ast - \alpha^\ast c\\
  \implies \alpha^\ast g^\ast(x) + \mu^\ast &= \tilde{\alpha} \tilde{g} (x) + \tilde{\mu}\\
  \tilde{g}(x) &\ne g^\ast (x)
\end{align*}
\item By the same reasoning, $\mu^\ast$ is not identified within the set of real numbers. To find a counterexample here, take
  \begin{align*}
    \tilde{\mu} &= \mu^\ast - \alpha^\ast t\\
    \tilde{g} (x) &= g^\ast (x) + t\\
    \tilde{\alpha} &= \alpha^\ast
  \end{align*}
\item \[
\sigma^\ast = \Var(Y \mid X = x)
\]
We observe $Y$ and $X$, so $(\sigma^\ast)^2$ is identified.
\item With $\alpha^\ast = 1, \mu^\ast = 0$
  \[
\E[Y \mid X = x] = g^\ast(x),
\]
so $g^\ast$ is identified. $(\sigma^\ast)^2$ is still identified because we still observe $X$ and $Y$.
\item In this case, $\mu$ is identified but $\alpha$ and $g$ are not. To see that $\mu^\ast$ is identified, consider $\tilde{\alpha}, \tilde{g}(x)$ and $\tilde{\mu}$ such that
  \[
\alpha^\ast g^\ast(x) + \mu^\ast = \tilde{\alpha}\tilde{g}(x) + \tilde{\mu}.
\]
\begin{align*}
  \underbrace{\tilde{\alpha}\tilde{g}(\overline{x})}_{0} + \tilde{\mu} &= \underbrace{\alpha^\ast g^\ast(\overline{x})}_0 + \mu^\ast\\
  \implies \tilde{\mu} &= \mu^\ast,
\end{align*}
so $\mu^\ast$ is identified. To see that $\alpha$ and $g$ are not identified, take
\begin{align*}
  \tilde{\alpha} &\equiv \frac{1}{c} \alpha^\ast\\
  \tilde{g} (x) &\equiv g^\ast(x) c
\end{align*}
\end{enumerate}
\end{document}
%%% Local Variables:
%%% mode: latex
%%% TeX-master: t
%%% End:
