 \documentclass{article}
 \title{202c Notes}
 \author{Christopher Ackerman}
 \usepackage{amsthm}
 \usepackage{url}
 % \usepackage[margin=.5in]{geometry}
 \usepackage{hyperref}
 \usepackage[dvipsnames]{xcolor}
 \usepackage{booktabs}
 \usepackage{enumitem}
 \newtheorem*{definition}{Definition}
 \newtheorem*{example}{Example}
 \newtheorem*{theorem}{Theorem}
 \newtheorem*{corollary}{Corollary}
 \newtheorem*{exercise}{Exercise}
 \newtheorem*{problem}{Problem}
 \newtheorem{question}{Question}
 \newcommand{\gr}{\textcolor{ForestGreen}}
 \newcommand{\rd}{\textcolor{red}}
 \newcommand{\R}{\mathbb{R}}
 \newcommand{\p}{\mathbb{P}}
 \newcommand{\E}{\mathbb{E}}
 \newcommand{\frall}{\ \forall}
 \newcommand{\st}{_{s_t}}
 \newcommand{\var}{\operatorname{Var}}
 \newcommand{\cov}{\operatorname{Cov}}
 \newcommand{\cor}{\operatorname{Cor}}


 \begin{document}
 \maketitle
 \newpage
 \section{Gravity Equations}
 The purpose of this portion of the class is to learn about a bunch of models that generate gravity equations. Gravity equations are an empirical regularity, and a bunch of economists provided microfoundations to explain where these equations come from. First let's define a gravity equation. Then, we'll explore a series of increasingly complex/realistic models that generate these equations. The first few models will be isomorphic to each other, meaning that we end up with exactly the same equations regardless of which model we start with.

 \begin{definition}[Gravity Equation]
   
 \end{definition}
 \rd{Review when we can just look at changes, and when we need to re-solve the entire model in levels to figure out what changes.}
\end{document}
